% This is a LaTeX file. It is a text file that is compiled
% by a program called LaTeX into a pretty PDF file.
% If you're viewing this file in Overleaf, you'll see that PDF
% in the window to the right.
%
% This file is for typesetting YOUR ANSWERS to this homework assignment.
% The LaTeX macro language is complicated, so we have inserted lots of
% documenting comments into the file. Comments start with '%'
% and continue to the end of the line. In Overleaf's edit window, they
% are colored green.
%
% Comments prefixed with 'Student:' are relevant to students. Skip any-
% thing else you don't understand, or ask me.

\documentclass{article}
%% This is some font management depending on the TeX “engine” being used.
%% Nothing to worry about.
\usepackage{ifxetex}
\ifxetex
  \usepackage{fontspec}
\else
  \usepackage[T1]{fontenc}
  \usepackage[utf8]{inputenc}
  \usepackage{lmodern}
\fi

%% Student: These lines describe some document metadata.

\title{Homework 1}

\usepackage{etoolbox}
\makeatletter\preto{\@title}{Answers to }\makeatother
\author{%
%% Student: change the next line to your name!
    Hongyi Zheng
\\
CS-UY 4533 Interactive Computer Graphics
}

%% These lines set up the question, answer, and solution environments.
\usepackage{amsthm}
\usepackage{amssymb}
\usepackage{amsmath}
\usepackage{siunitx}
\usepackage{graphicx}
\usepackage{enumitem}
\DeclareSIUnit{\pound}{lb}

\theoremstyle{plain}
\newtheorem{question}{Question}

\newenvironment{answer}[1][Answer]
    {\begin{proof}[#1]\renewcommand\qedsymbol{$\vartriangle$}}
    {\end{proof}}
\newenvironment{solution}[1][Solution]
    {\begin{proof}[#1]\renewcommand\qedsymbol{$\blacktriangle$}}
    {\end{proof}}
\makeatletter
    \newcommand{\stepenumdepth}{\advance\@enumdepth\@ne}
\makeatother
\AtBeginEnvironment{question}{\stepenumdepth}
\AtBeginEnvironment{answer}{\stepenumdepth}
\AtBeginEnvironment{solution}{\stepenumdepth}


\usepackage{hyperref}
%% This is the beginning of the part of the file that describes
%% the actual text of the document.
%% That's why it says `\begin{document}' below. :-)
\begin{document}
\maketitle

\begin{enumerate}[label=\textbf{(\alph*)}]
\item
The first midpoint to check is the midpoint between first $N=(r, 1)$ and $NW = (r-1, 1)$, which is $(r - \frac{1}{2}, 1)$. To avoid floating-point computations, we define $F(x, y) = 4(x^2 + y^2 -r^2)$. Thus, $D_{start} = F(M_{start}) = 5- 4r$. \\

Then, we follow those steps to update $D$:
\begin{equation*}
    D_{new} = D_{old} + F(M_{new}) - F(M_{old})
\end{equation*}
If $N$ is chosen $(D_{old} < 0)$,
\begin{equation*}
\begin{aligned}
D_{new} &=D_{old}+4\left(x^{2}+(y+1)^{2}-r^{2}\right)-4\left(x^{2}+y^{2}-r^{2}\right) \\
&=D_{old} +8y+4
\end{aligned}
\end{equation*}
Here, $y$ is the $y-$coordinate of the $M_{old}$, and is also the $y-$coordinate of the chosen $N$. \\

If $NW$ is chosen $(D_{old} \geq 0)$,
\begin{equation*}
\begin{aligned}
D_{new} &=D_{old}+4\left((x-1)^{2}+(y+1)^{2}-r^{2}\right)-4\left(x^{2}+y^{2}-r^{2}\right) \\
&=D_{old} +8y - 8x + 8
\end{aligned}
\end{equation*}
Here, $y$ is the $y-$coordinate of the $M_{old}$, and is also the $y-$coordinate of the chosen $NW$. However, $x = x_{M_{old}} = x_{NW} + \frac{1}{2}$
Therefore,
\begin{equation*}
\begin{aligned}
D_{new} &=D_{old} +8y - 8x + 8 \\
&= D_{old} + 8y_{NW} - 8(x_{NW} + \frac{1}{2}) + 8 \\
&= D_{old} + 8y_{NW} - 8x_{NW} + 4
\end{aligned}
\end{equation*}

In conclusion,
\begin{equation*}
    D_{new} =
    \begin{cases}
        D_{old} + 8y_N + 4 & (D_{old} < 0) \\
        D_{old} + 8y_{NW} - 8x_{NW} + 4 & (D_{old} \geq 0)
    \end{cases}
\end{equation*}

\item
In OpenGL's screen coordinate system, the origin is on the bottom-left corner, the $x-$axis increase to the right and the $y-$axis increase to the top. It is not possible to display pixels at $(x, y)$ where $x < 0$ or $y < 0$;
\end{enumerate}

\end{document}
\endinput
%%
%% End of file `hw04.ans.tex'.
